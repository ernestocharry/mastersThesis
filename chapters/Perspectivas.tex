\chapter{Perspectivas}
\lettrine{E}{}n primer lugar, la necesidad de realizar la corrección por densidad y composición en todas las muestras implica la integración de toda la información necesaria de forma operativa. Ello no es fácil debido a la diversidad de las técnicas utilizadas, la heterogeneidad de los resultados emitidos por los equipos y la necesidad de la intervención humana en cada uno de los pasos. 
\\
\\
Si bien se trata de un esfuerzo importante y diverso en la mayoría de los laboratorios, se propone la siguiente metodología:
\\ \\
\textbf{1.}  Crear un sistema de codificación única y completa de las muestras, con un sistema automático de detección de errores.
\\ 
\textbf{2.} Para cada uno de los equipos de análisis elemental, definición del producto final (concentraciones) en unidades absolutas, y acumulación de la información final en un servidor único. 
\\ 
\textbf{3.} En el caso de la espectrometría de rayos gamma, generar calibraciones de referencia únicas para cada detector y geometría utilizadas. No es necesario definir la densidad final de cada muestra.
\\ 
\textbf{4.} Al final (o durante) del proceso analítico, capturar de forma automática toda la información disponible de las muestras.
\\ 
\textbf{5.} Con esta información, utilizar un código para realizar la corrección de densidad y autoabsorción mediante ANGLE (o un código similar) y proporcionar las actividades corregidas de los radionúclidos de interés.
\\ \\
Una posibilidad a explorar es generar una extensa base de datos en la cual, dada la concentración de un número limitado de elementos y densidades, se pueda interpolar la corrección de eficiencia para cada muestra. Además, este ejercicio debería ser repetido para cada combinación geometría – detector de interés. Antes de abordar un proyecto de estas dimensiones, sería necesario realizar una estimación del esfuerzo de cómputo necesario y valorar el costo-beneficio del mismo. Otra posible alternativa es investigar sobre la posibilidad de utilizar modelos simplificados para realizar la estimación de la eficiencia corregida para cada muestra.
\\
\\
Una de las limitaciones de este trabajo es la necesidad de realizar aproximaciones de las concentraciones de H y O. Al menos en el caso del oxígeno, sería deseable explorar técnicas analíticas que permitan determinar este elemento de forma sistemática. Por ejemplo, el sistema \textit{Vario-Cube} de Elementar tiene esta capacidad, y se propone explorar el costo beneficio del uso de esta técnica, pues por un lado la corrección final sería menor al 3 \% y el costo del equipo y del recurso humano puede ser alto para los laboratorios que no lo tengan implementado. 