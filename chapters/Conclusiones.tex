\chapter{Conclusiones}
\lettrine{L}{}a eficiencia de detección para \PbCero\, en los detectores utilizados en este trabajo fue del 63 al 71 \%, lo cual confirma la utilidad de la configuración de pozo para medir pequeñas muestras de material. La corrección de eficiencia por autoabsorción de cada muestra requiere el conocimiento del 100 \% de su composición elemental. El análisis elemental de los núcleos sedimentarios EU-VIII, GOMRI-500, PCm, LTAF, SAMO-14-2 y TEHUA-XII mediante la espectrometría de fluorescencia de rayos X y el análisis elemental de carbono - nitrógeno permitió constatar la gran variabilidad en la composición elemental inter- e intra-núcleos, de los que se conoce al menos el 46 \% de su composición promedio.  
\\
\\
El resto de la composición (composición desconocida) fue atribuida principalmente al H (componente mayoritario de la materia orgánica) y el oxígeno (componente mayoritario, entre otros, de los carbonatos). La diferencia promedio de utilizar del 0 al 100 \% de oxígeno para corregir la eficiencia fue de 3 \%, que es menor que la incertidumbre típica de las medidas de espectrometría de rayos gamma. Se concluyó que se podía asumir una composición desconocida con el 50 \% de contribución de oxígeno y el 50 \% de hidrógeno. 
\\
\\
La diferencia entre la eficiencia de referencia y corregida de todas las muestras varió entre 1.7 y 16 \%. Las menores diferencias fueron observadas en muestras con alta concentración de C, y por lo tanto de materia orgánica, con bajo coeficiente de atenuación de radiación gamma. Se concluyó que en los casos de alto contenido de materia orgánica, la corrección por autoabsorción no es relevante. Este fue el caso del núcleo sedimentario procedente de un manglar PCm, que presentó el mayor porcentaje de composición conocida promedio (75 \%), una diferencia promedio de 1.7 \% entre las eficiencias de referencia y corregida, y una diferencia próxima a cero entre los modelo de fechado con una composición de referencia y corregida. En este caso, no sería necesario realizar la corrección de la eficiencia por composición de la muestra. Sin embargo, siempre es importante corregir por la densidad de la muestra que puede causar desviaciones entre 0.11 hasta 0.43 g cm$^{-3}$. Aproximadamente, las desviaciones de las actividades de \PbCero\, en los núcleos sedimentarios fueron de 10 \% para EU-VIII, LTAF, SAMO-14-2 y TEHUA-XII, y de 16 \% para GOMRI-500 para composiciones de referencia y corregidas. 
\\
\\
Las desviaciones máximas observadas de las actividades corregidas (16 \%) no son inesperadas, pero sí importantes, y muestran que omitir la corrección por densidad y autoabsorción puede provocar valores de actividad de \PbCero\, demasiado bajos. Esta observación es relevante para todas las medidas de \PbCero\, (y otros radionúclidos) donde el valor absoluto de la actividad es crítico, como pueden ser los planes de vigilancia ambiental de la industria del ciclo del combustible nuclear (incluyendo la minería) y la presencia de actividades elevadas de elementos radiactivos naturales, bien sea por causas antropogénicas o no (conocidos como NORM). Esta conclusión subraya la necesidad de calibrar los sistemas de espectrometría de rayos gamma siguiendo los máximos criterios de calidad posible, y en particular la necesidad de aplicar de forma sistemática correcciones de densidad y autoabsorción, especialmente para los emisores gamma de baja energía. 
\\
\\
La cuantificación precisa de la actividad del \PbCero\, es también importante en el fechado de núcleos sedimentarios con \PbCero. Si bien se puede dar el caso de que la corrección de eficiencia sea relativamente constante y afecte poco al modelo de edad (por ejemplo, 0.1 \% para el núcleo sedimentario PCm), sí que es importante conocer con la mayor exactitud posible la profundidad a partir de la cual se puede asumir que se ha llegado al equilibrio secular. Las correcciones realizadas en este trabajo permitieron concluir que todos los núcleos sedimentarios habían llegado al equilibrio secular, condición imprescindible para realizar el fechado con el modelo CF (flujo de \PbCeroEx\, constante). 
\\
\\
Con la información disponible, incluyendo las actividades de \PbCero\, corregidas y de referencia, se realizó el fechado de los núcleos sedimentarios utilizando el modelo CF. Las fechas corregidas variaron entre 0.1 \% y 22 \%, demostrando la necesidad del uso de la corrección de la eficiencia para el fechado del \PbCero. La corrección media de los modelos de edad fue de 3 \% para GOMRI-500, 2.5 \% para TEHUA-XII, 4 \% para LTAF, 0.1 \% para PCm, 22 \% para EU-VIII y 2 \% para SAMO-14-2. Estas desviaciones medias no se correlacionaron de forma simple con la corrección de la eficiencia (y por lo tanto de la actividad) ni con su variabilidad, pues el modelo de edad incluye de forma compleja las sumas de los productos de las masas y las concentraciones. Por lo tanto, no se puede anticipar de forma simple en qué casos estas correcciones son relevantes y, por lo tanto, es necesario realizar la corrección de la densidad y eficiencia en todos los casos para obtener fechados más confiables.