%\cleardoublepage
%\phantomsection
\addcontentsline{toc}{chapter}{Resumen}
\chapter*{Resumen}
\lettrine{E}{}l radionúclido natural \PbCero\, es utilizado comúnmente para fechar sedimentos recientes ($\leq$ 100 años) y reconstruir la historia del Cambio Global en ecosistemas acuáticos. Su cuantificación se realiza usualmente mediante espectrometría de rayos gamma de alta resolución y bajo fondo, a través de la detección de su rayo gamma cuya energía es 46.54 keV que, al encontrarse en la zona de bajas energías del espectro, está seriamente afectado por la autoabsorción en la muestra. Por lo tanto, la correcta cuantificación de \PbCero\, en muestras de sedimentos requiere realizar correcciones por densidad y composición de la muestra. 
\\ \\
La composición elemental de algunas secciones de seis núcleos sedimentarios, procedentes de ecosistemas acuáticos contrastantes de México, se determinó mediante el análisis de la concentración de elementos mayoritarios, minoritarios y trazas por  espectrometría de fluorescencia de rayos X (XRF) y de la concentración de carbono y nitrógeno (C-N) por medio de analizador elemental. El 100 \% de la composición se completa con una fracción no medida de oxígeno – hidrógeno. Los elementos mayoritarios fueron C, N, Na, Mg, Al, Si, S, Cl, K, Ca y Fe. Las concentraciones elementales variaron inter- e intra-núcleos. La suma promedio de las concentraciones de la composición elemental medida fue $\sim$ 50 \% para todas las muestras.
\\ \\
Con un patrón de emisores monoenergéticos en medio acuoso, se construyó una curva de calibración para los sistemas de espectrometría gamma que se utilizó para corregir la eficiencia de \PbCero, así como del radionúclido \PbCuatro, debido a la composición y densidad de los sedimentos. Las desviaciones de la actividad corregida respecto a la no corregida estuvieron en el intervalo de 2 a 16 \%, y se concluyó que la actividad de la muestra requiere la corrección propuesta en la mayor parte de los casos estudiados. Para comprobar la validez de la corrección realizada, en el marco del fechado de núcleos con \PbCero, se seleccionaron secciones donde se espera el equilibrio secular entre el \PbCero\, y el \PbCuatro\, (a su vez en equilibrio con el \Ra) y se comprobó que la desviación media respecto al equilibrio fue de $7 \pm 8$ \%. Se realizó el fechado de los núcleos sedimentarios con \PbCero\, mediante el modelo de flujo constante. Se concluyó que los resultados del fechado dependen de una forma compleja de los valores absolutos y de la variabilidad de las eficiencias corregidas y las masas de las secciones a lo largo del núcleo. 
\newpage
\textbf{Palabras claves:} \PbCero, fechado de sedimentos, espectrometría de rayos gamma, eficiencia, composición elemental, densidad