\cleardoublepage
\phantomsection
\addcontentsline{toc}{chapter}{Abreviaciones y símbolos}
\chapter*{Abreviaciones}
\begin{table}[h]
\centering 
\begin{tabular}{|r|l|} \hline
\rowcolor{Blue2} Abreviación & Nombre extenso \\ \hline
\rowcolor{Blue1} LGIG: & Laboratorio de Geoquímica Isotópica y Geocronología \\
\rowcolor{Blue1} & Instituto de Ciencias del Mar y Limnología \\
\rowcolor{Blue1} & Universidad Nacional Autónoma de México \\
\rowcolor{Blue1} & Unidad Académica Mazatlán, Sinaloa, México.  \\
\rowcolor{Blue1} XRF: & Fluorescencia de rayos X \\
\rowcolor{Blue1} C-N: & Carbono - nitrógeno \\ 
\rowcolor{Blue1} MDA: & Actividad mínima detectable (MDA, por sus siglas en inglés)  \\
\rowcolor{Blue1} LME: & Grandes ecosistemas marinos (LME, por sus siglas en inglés) \\ 
\rowcolor{Blue1} FWHM: & Anchura a media altura (FWHM, por sus siglas en inglés) \\ 
\rowcolor{Blue1} TCSC: & Corrección por coincidencia de suma (TCSC, por sus siglas en inglés) \\ 
\rowcolor{Blue1} ROI: & Región de interés (ROI, por sus siglas en inglés) \\ 
\rowcolor{Blue1} Dif.:& Diferencia \\ 
\rowcolor{Blue1} MAR: & Tasa de acumulación másica (MAR, por sus siglas en inglés) \\
\rowcolor{Blue1} SAR: & Tasa de acumulación sedimentaria (SAR, por sus siglas en inglés) \\ 
\rowcolor{Blue1} CF: & Modelo de fechado de Flujo Constante (CF, por sus siglas en inglés) \\
\rowcolor{Blue1} MRC: & Material de Referencia Certificado\\
\rowcolor{Blue1} Corr :& Corregido \\
\rowcolor{Blue1} Ref :& Referencia\\
\rowcolor{Blue1} NA: &  No aplica \\
\rowcolor{Blue1} $^{210}$Pb$_\text{base}$: &  \PbCero\, producido \textit{in situ} (en equilibrio con \Ra) \\ 
\rowcolor{Blue1} \PbCeroEx: & \PbCero\, en exceso \\
\hline
\end{tabular}
\end{table}

\begin{table}[h]
\centering 
\begin{tabular}{|r|l|} \hline
\rowcolor{Blue2} \multicolumn{2}{|c|}{\textbf{Referentes a núcleos sedimentarios}}  \\ \hline
\rowcolor{Blue2} Abreviación & Nombre extenso \\ \hline
\rowcolor{Blue1} LTAF & Laguna de Términos, Atasta Franja, Golfo de México, Campeche.\\
\rowcolor{Blue1} EU-VIII & Estero de Urias, Pacífico, Sinaloa, México. \\
\rowcolor{Blue1} GOMRI-500 & Golfo de México sur. \\
\rowcolor{Blue1} PCm & Punta Caracol manglar, mar Caribe, Quintana Roo, México. \\
\rowcolor{Blue1} Lago SAMO-14-2 & Santa María del Oro, Nayarit, México. \\
\rowcolor{Blue1} TEHUA-XII & Golfo de Tehuantepec, Pacífico Mexicano\\ \hline 
\end{tabular}
\end{table}

\begin{table}[h]
\centering 
\begin{tabular}{|c|c|c|} \hline
\rowcolor{Blue2} Símbolo & Unidades & Significado \\ \hline
\rowcolor{Blue1} $\mu$ & [longitud$^{-1}$] & Coeficiente lineal de atenuación  \\
\rowcolor{Blue1} $\dfrac{\mu}{\rho}$ & [longitud$^2$ masa$^{-1}$]  & Coeficiente másico de atenuación \\
\rowcolor{Blue1} $m_e$ & [masa] & Masa del electrón \\
\rowcolor{Blue1} $c$ & [longitud tiempo$^{-1}$] & Velocidad de la luz \\
\rowcolor{Blue1} $\epsilon$ & & Eficiencia de detección \\ 
\rowcolor{Blue1} $T_{\frac{1}{2}}$ &  [tiempo] & Semiperiodo de desintegración \\
\hline
\end{tabular}
\end{table}