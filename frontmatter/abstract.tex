\cleardoublepage
\phantomsection
\addcontentsline{toc}{chapter}{Abstract}
\chapter*{Abstract}
\lettrine{T}{}he natural radionuclide \PbCero\, is commonly used to date recent sediments ($\leq$ 100 years) and reconstruct the history of Global Change in multiple aquatic ecosystems. Its quantification is usually carried out  by high-resolution low-background gamma spectrometry, through the detection of its gamma ray 46.54 keV which, being in the low energy region of the spectrum, is seriously affected by sample self-absorption. Therefore, the correct quantification of \PbCero\, in sediment samples requires corrections for sample density and composition.
\\
\\
The elemental composition of some sections of six sedimentary cores, from contrasting aquatic ecosystems in Mexico, was determined by analyzing the concentration of majority, minor and trace elements by X-ray fluorescence spectrometry (XRF) and the concentration of carbon and nitrogen (C-N) by elemental analyzer and, the estimation of an unmeasured fraction of oxygen - hydrogen. The major elements were C, N, Na, Mg, Al, Si, S, Cl, K, Ca and Fe. The elemental concentrations varied inter- and intra-cores. The average sum of the concentrations of the measured elemental composition was $\sim$ 50 \%. for all samples.
\\
\\
With a pattern of monoenergetic emitters in aqueous media, a calibration curve was constructed that was used to correct the efficiency of \PbCero\, and \PbCuatro\, due to the composition and density of the sediments. The deviations of the corrected activity versus uncorrected were in the range 2 - 16 \%, and it was concluded that the activity of the sample requires the proposed correction in most of the studied cases. To verify the validity of the correction in the framework of \PbCero\, sediment dating, sections were selected where the secular equilibrium between \PbCero\, and \PbCuatro\, (in turn in equilibrium with \Ra) was expected, and it was found that the mean deviation from equilibrium was $7 \pm 8$ \%. Dating of sedimentary cores was performed  with \PbCero\, by using the constant flow model. It was concluded that the results of the dating depend on a complex form of the absolute values and the variability of the corrected efficiencies and the masses of the sections along the core. 
\newpage
\textbf{Keywords:} \PbCero, sediment dating, gamma ray spectrometry, efficiency, elemental composition, density