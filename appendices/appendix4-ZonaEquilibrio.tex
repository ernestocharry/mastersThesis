\chapter{Secciones en equilibrio secular}\label{ApexZonaEquilibrio}
\lettrine{D}{}e acuerdo a las Figuras \ref{Fig-EUVIII-Comp}, \ref{Fig-GOMRI500-Comp}, \ref{Fig-PCm-Comp}, \ref{Fig-LTAF-Comp}, \ref{Fig-SAMO142-Comp} y \ref{Fig-TEHUAXII-Comp} se espera que en las últimas secciones de los núcleos sedimentarios exista un equilibrio secular entre \PbCero\, y \PbCuatro. En la Figura \ref{Fig-SeccEquilibrioSec} se muestra las actividades de estos radionúclidos de interés y se considera que existe un equilibrio secular en una sección cuando ambas actividades se sobreponen al considerar las incertidumbres. Estas secciones aparecen en verde en la Figura \ref{Fig-SeccEquilibrioSec}. La falta de equilibrio secular para las secciones restantes puede atribuirse a razones geoquímicas (por ejemplo, movilidad de \Ra\, en condiciones altamente reductoras), pero no debido a la corrección por composición elemental en el análisis de espectrometría de rayos gamma.
\begin{figure}
\centering
\includegraphics[width=0.9\textwidth]{Imagenes/Analisis_50Oxigeno.png}
\caption{secciones pertenecientes a la zona de equilibrio de los núcleos sedimentarios. Con $^{*}$: secciones en equilibrio secular debido a la superposición de las actividades de \PbCero\, y \PbCuatro.}\label{Fig-SeccEquilibrioSec}
\end{figure}